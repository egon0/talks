\section{Plugins}
\begin{frame}
  \frametitle{Plugins für Geany}
  \begin{block}{}
    \begin{itemize}
    \item Eine große Anzahl der Plugins werden über ein gemeinsames
      Paket angeboten
    \item Derzeit in C/C++ direkt unterstützt; Lua Plugins über GeanyLUA
      möglich; Python in Entwicklung
    \item Einfaches Plugin mit ca. 35 Zeilen Code möglich
    \item Möglichkeit Callbacks für Events zu registrieren. Z.B.
      \begin{itemize}
        \item User tippt
        \item Dokument wird geöffnet/geschlossen/gespeichert
      \end{itemize}
    \item Zugriff auf alle wichtigen Elemente von Geany via API:
      \begin{itemize}
        \item Werkzeugleiste
        \item Seitenleiste
        \item Dateimenü
        \item Editorkomponente
      \end{itemize}
    \end{itemize}
  \end{block}
\end{frame}

\begin{frame}
  \frametitle{Plugins für Geany}
  \begin{block}{Plugins}
    \begin{itemize}
    \item \textbf{Devhelper} -- Komfortabler Zugriff auf Dokumentation
    \item \textbf{GeanyVC} - Anbindung an populäre
        Versions\-ver\-waltungs\-systeme wie git, svn, svk, cvs
    \item \textbf{GeanyLaTeX} -- Unterstützung bei Erstellung von
        \LaTeX-Dokumenten
    \item \textbf{Spellcheck} -- Rechtschreibprüfung
    \item \textbf{GeanyPrj} -- Erweiterter Projektsupport
    \item \textbf{XML PrettyPrinter} -- Reformatierung von XML-Dokumenten
    \item \textbf{Addons} -- Generisches Plugin mit verschiedenen,
          kleinen Hilfsfunktionen
    \item \textbf{WebHelper} -- Ein Plugin mit integrierten Browser
          und HTML-Analyse.
    \end{itemize}
  \end{block}
\end{frame}

\begin{frame}
  \frametitle{GeanyVC - Die Anbindung an svn \& Co}
  \begin{block}{}
    \begin{itemize}
    \item Unterstützt:
      \begin{itemize}
      \item Subversion (svn)
      \item git
      \item Bazaar (bzr)
      \item CVS
      \item Mercurial (hg)
      \item svk
      \end{itemize}
    \item Anbindung verschiedene grundiegende Tätigkeiten wie commit,
      add, log, blame/annotate
    \item Steuerung über Tastaturbefehle möglich
    \item Rechtschreibprüfung im Commit-Nachrichtendialog
    \end{itemize}
  \end{block}
\end{frame}

\begin{frame}
  \frametitle{Geany\LaTeX{} - \LaTeXe mit Geany}
  \begin{columns}[c]
    \column[c]{1.5cm} \huge \LaTeX
    \column{8.5cm} \normalsize
    \begin{block}{Fähigkeiten}
      \begin{itemize}
      \item Einfügen von Sonderzeichen und Umwandlung von
        Sonderzeichen in \TeX-Schreibweise beim Tippen
      \item Unterstützung beim Einfügen von Lesezeichen, Verweisen,
        Befehlen, Umgebungen und bibliographischen Einträgen
      \item Formatierungen
      \item \LaTeX-Assistent zum schnellen Erstellen von Dokumenten
      \item Werkzeugliste mit wichtigsten Formatierungen
      \item Unterstützung bei einfachen BibTeX-Aufgaben
      \end{itemize}
    \end{block}
  \end{columns}
\end{frame}

\begin{frame}
  \frametitle{GeanySendMail}
  \begin{block}{}
    \begin{itemize}
      \item Plugin um geöffnete Dateien per Mail zu versenden
      \item Unterstützung für
        \begin{itemize}
        \item Frei definierbarer Programmaufruf mit Platzhaltern für
          Dateinamen und Empfänger Mailadresse
        \item Speicherung der letzten Zieladresse
        \end{itemize}
    \end{itemize}
  \end{block}
\end{frame}

\begin{frame}
  \frametitle{Spellcheck - Die Rechtschreibprüfung}
  \begin{block}{}
    \begin{itemize}
    \item Überprüfung während des Tippens oder für das komplette
      Dokument
    \item Vorschläge zur Rechtschreibung innerhalb des Kontextmenüs
    \item Kann auf alle Rechtschreibsysteme zugreifen, die Enchant
      unterstützt wie aspell, ispell, wmspell.
    \end{itemize}
  \end{block}
\end{frame}

\begin{frame}
    \frametitle{XML PrettyPrinter}
    \begin{block}{}
    \begin{itemize}
        \item Unterstützt bei der Formatierung von XML-Dokumenten
              in eine menschenfreundliche Form durch einfügen von
              Leerzeichen und Zeilenumbrüchen basierend auf dem DOM
        \item Basiert auf libXML2
        \item Schnelles Formatieren des XML-Dokumentes
    \end{itemize}
    \end{block}
\end{frame}

\begin{frame}
    \frametitle{Devhelp - Die GNOME Devhelp Anbindung}
    \begin{block}{}
        \begin{itemize}
            \item Bietet ein Interface zu Devhelp in Geany
            \item Unterstützt auch Google code und Manual-Seiten
            \item Einfacher Wechsel zwischen Dokumentation und Code
            \item Gewohnte Devhelp-Navigation (Suche, manuelle navigieren
                  durch Handbücher) in der Seitenleiste
        \end{itemize}
    \end{block}
\end{frame}

\begin{frame}
    \frametitle{Addons -- Das generische Plugin}
    \begin{block}{}
        \begin{itemize}
            \item Bietet eine Reihe von nützlichen Erweiterungen an
            \begin{itemize}
                \item \textbf{XML Tagging:} Ermöglicht das einfache Einfügen
                      einer Markierung in XML-Tags
                \item \textbf{Open URI:} Öffnet einen aktuell markierten
                      Text im Browser
                \item \textbf{Tasks:} Durchsucht Dokumente nach Markerungen
                      wie »TODO« oder »FIXME« und listet sie für einfach
                      Zugriff auf
                \item \textbf{Systray:} Fügt ein Geany-Symbol in die
                      Systemleiste ein.
                \item \textbf{Bookmark List}: Bietet eine erweitere
                      Lesezeichenübersicht für die aktuelle Datei an
                \item ...
            \end{itemize}
        \end{itemize}
    \end{block}
\end{frame}
