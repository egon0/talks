\section{Plugins}
\begin{frame}
	\frametitle{Plugins für Geany}
	\begin{block}{}
		\begin{itemize}
			\item Derzeit in C direkt unterstützt; Lua Plugins über
				  GeanyLUA möglich
			\item Einfaches Plugin mit ca. 35 Zeilen Code möglich
			\item Möglichkeit Callbacks für Events zu registrieren. Z.B.
				\begin{itemize}
					\item User tippt
					\item Dokument wird geöffnet/geschlossen/gespeichert
				\end{itemize}
			\item Zugriff auf alle wichtigen Elemente von Geany wie
				\begin{itemize}
					\item Werkzeugleiste
					\item Seitenleiste
					\item Dateimenü
					\item Editorkomponente
				\end{itemize}
		\end{itemize}
	\end{block}
\end{frame}

\begin{frame}
	\frametitle{Plugins für Geany}
	\begin{block}{Plugins}
		\begin{itemize}
			\item \textbf{GeanyVC} - Anbindung an populäre
				Versions\-ver\-waltungs\-systeme wie git, svn, svk, cvs
			\item \textbf{GeanyLaTeX} - Unterstützung bei Erstellung von
				\LaTeX-Dokumenten
			\item \textbf{Spellcheck} - Rechtschreibprüfung
			\item \textbf{GeanyPrj} - Erweiterter Projektsupport
			\item \textbf{GeanyDoc} - Komfortabler Zugriff auf Dokumentation
		\end{itemize}
	\end{block}
\end{frame}

%\section{Plugins im Detail}
%\begin{frame}
	%\frametitle{GeanyVC - Die Anbindung an svn \& Co}
	%\begin{block}{}
		%\begin{itemize}
			%\item
		%\end{itemize}
	%\end{block}
%\end{frame}

\begin{frame}
	\frametitle{GeanyLaTeX - \LaTeXe mit Geany}
	\begin{columns}[c]
		\column[c]{2cm}
			\huge \LaTeX
		\column{8cm}
			\begin{block}{}
				\begin{itemize}
					\item Einfügen von Sonderzeichen und Umwandlung von
						  Sonderzeichen in \TeX-Schreibweise beim Tippen
					\item Unterstützung beim Einfügen von Lesezeichen,
						  Verweisen und bibliographischen Einträgen
					\item Formatierungen
					\item \LaTeX-Assistent zum schnellen Erstellen von
						  Dokumenten
				\end{itemize}
			\end{block}
	\end{columns}
\end{frame}

\begin{frame}
	\frametitle{GeanySendMail}
	\begin{itemize}
		\item Plugin um geöffnete Dateien per Mail zu versenden
		\item Unterstützung für
			\begin{itemize}
				\item Frei definierbarer Programmaufruf mit Platzhaltern
					  für Dateinamen und Empfänger Mailadresse
				\item Speicherung der letzten Zieladresse
			\end{itemize}
	\end{itemize}
\end{frame}

%\begin{frame}
	%\frametitle{Spellcheck - Die Rechtschreibprüfung}
	%\begin{block}{}
		%\begin{itemize}
			%\item
		%\end{itemize}
	%\end{block}
%\end{frame}


